% Options for packages loaded elsewhere
\PassOptionsToPackage{unicode}{hyperref}
\PassOptionsToPackage{hyphens}{url}
%
\documentclass[
]{article}
\usepackage{amsmath,amssymb}
\usepackage{iftex}
\ifPDFTeX
  \usepackage[T1]{fontenc}
  \usepackage[utf8]{inputenc}
  \usepackage{textcomp} % provide euro and other symbols
\else % if luatex or xetex
  \usepackage{unicode-math} % this also loads fontspec
  \defaultfontfeatures{Scale=MatchLowercase}
  \defaultfontfeatures[\rmfamily]{Ligatures=TeX,Scale=1}
\fi
\usepackage{lmodern}
\ifPDFTeX\else
  % xetex/luatex font selection
\fi
% Use upquote if available, for straight quotes in verbatim environments
\IfFileExists{upquote.sty}{\usepackage{upquote}}{}
\IfFileExists{microtype.sty}{% use microtype if available
  \usepackage[]{microtype}
  \UseMicrotypeSet[protrusion]{basicmath} % disable protrusion for tt fonts
}{}
\makeatletter
\@ifundefined{KOMAClassName}{% if non-KOMA class
  \IfFileExists{parskip.sty}{%
    \usepackage{parskip}
  }{% else
    \setlength{\parindent}{0pt}
    \setlength{\parskip}{6pt plus 2pt minus 1pt}}
}{% if KOMA class
  \KOMAoptions{parskip=half}}
\makeatother
\usepackage{xcolor}
\usepackage[margin=1in]{geometry}
\usepackage{color}
\usepackage{fancyvrb}
\newcommand{\VerbBar}{|}
\newcommand{\VERB}{\Verb[commandchars=\\\{\}]}
\DefineVerbatimEnvironment{Highlighting}{Verbatim}{commandchars=\\\{\}}
% Add ',fontsize=\small' for more characters per line
\usepackage{framed}
\definecolor{shadecolor}{RGB}{248,248,248}
\newenvironment{Shaded}{\begin{snugshade}}{\end{snugshade}}
\newcommand{\AlertTok}[1]{\textcolor[rgb]{0.94,0.16,0.16}{#1}}
\newcommand{\AnnotationTok}[1]{\textcolor[rgb]{0.56,0.35,0.01}{\textbf{\textit{#1}}}}
\newcommand{\AttributeTok}[1]{\textcolor[rgb]{0.13,0.29,0.53}{#1}}
\newcommand{\BaseNTok}[1]{\textcolor[rgb]{0.00,0.00,0.81}{#1}}
\newcommand{\BuiltInTok}[1]{#1}
\newcommand{\CharTok}[1]{\textcolor[rgb]{0.31,0.60,0.02}{#1}}
\newcommand{\CommentTok}[1]{\textcolor[rgb]{0.56,0.35,0.01}{\textit{#1}}}
\newcommand{\CommentVarTok}[1]{\textcolor[rgb]{0.56,0.35,0.01}{\textbf{\textit{#1}}}}
\newcommand{\ConstantTok}[1]{\textcolor[rgb]{0.56,0.35,0.01}{#1}}
\newcommand{\ControlFlowTok}[1]{\textcolor[rgb]{0.13,0.29,0.53}{\textbf{#1}}}
\newcommand{\DataTypeTok}[1]{\textcolor[rgb]{0.13,0.29,0.53}{#1}}
\newcommand{\DecValTok}[1]{\textcolor[rgb]{0.00,0.00,0.81}{#1}}
\newcommand{\DocumentationTok}[1]{\textcolor[rgb]{0.56,0.35,0.01}{\textbf{\textit{#1}}}}
\newcommand{\ErrorTok}[1]{\textcolor[rgb]{0.64,0.00,0.00}{\textbf{#1}}}
\newcommand{\ExtensionTok}[1]{#1}
\newcommand{\FloatTok}[1]{\textcolor[rgb]{0.00,0.00,0.81}{#1}}
\newcommand{\FunctionTok}[1]{\textcolor[rgb]{0.13,0.29,0.53}{\textbf{#1}}}
\newcommand{\ImportTok}[1]{#1}
\newcommand{\InformationTok}[1]{\textcolor[rgb]{0.56,0.35,0.01}{\textbf{\textit{#1}}}}
\newcommand{\KeywordTok}[1]{\textcolor[rgb]{0.13,0.29,0.53}{\textbf{#1}}}
\newcommand{\NormalTok}[1]{#1}
\newcommand{\OperatorTok}[1]{\textcolor[rgb]{0.81,0.36,0.00}{\textbf{#1}}}
\newcommand{\OtherTok}[1]{\textcolor[rgb]{0.56,0.35,0.01}{#1}}
\newcommand{\PreprocessorTok}[1]{\textcolor[rgb]{0.56,0.35,0.01}{\textit{#1}}}
\newcommand{\RegionMarkerTok}[1]{#1}
\newcommand{\SpecialCharTok}[1]{\textcolor[rgb]{0.81,0.36,0.00}{\textbf{#1}}}
\newcommand{\SpecialStringTok}[1]{\textcolor[rgb]{0.31,0.60,0.02}{#1}}
\newcommand{\StringTok}[1]{\textcolor[rgb]{0.31,0.60,0.02}{#1}}
\newcommand{\VariableTok}[1]{\textcolor[rgb]{0.00,0.00,0.00}{#1}}
\newcommand{\VerbatimStringTok}[1]{\textcolor[rgb]{0.31,0.60,0.02}{#1}}
\newcommand{\WarningTok}[1]{\textcolor[rgb]{0.56,0.35,0.01}{\textbf{\textit{#1}}}}
\usepackage{graphicx}
\makeatletter
\def\maxwidth{\ifdim\Gin@nat@width>\linewidth\linewidth\else\Gin@nat@width\fi}
\def\maxheight{\ifdim\Gin@nat@height>\textheight\textheight\else\Gin@nat@height\fi}
\makeatother
% Scale images if necessary, so that they will not overflow the page
% margins by default, and it is still possible to overwrite the defaults
% using explicit options in \includegraphics[width, height, ...]{}
\setkeys{Gin}{width=\maxwidth,height=\maxheight,keepaspectratio}
% Set default figure placement to htbp
\makeatletter
\def\fps@figure{htbp}
\makeatother
\setlength{\emergencystretch}{3em} % prevent overfull lines
\providecommand{\tightlist}{%
  \setlength{\itemsep}{0pt}\setlength{\parskip}{0pt}}
\setcounter{secnumdepth}{-\maxdimen} % remove section numbering
\ifLuaTeX
  \usepackage{selnolig}  % disable illegal ligatures
\fi
\usepackage{bookmark}
\IfFileExists{xurl.sty}{\usepackage{xurl}}{} % add URL line breaks if available
\urlstyle{same}
\hypersetup{
  pdftitle={Reproducable Research and Figures Assignment},
  hidelinks,
  pdfcreator={LaTeX via pandoc}}

\title{Reproducable Research and Figures Assignment}
\author{}
\date{\vspace{-2.5em}2024-12-11}

\begin{document}
\maketitle

\section{Reproducible Research and Figures
Assignment}\label{reproducible-research-and-figures-assignment}

\subsection{Preparatory Steps}\label{preparatory-steps}

Loading libraries

\begin{Shaded}
\begin{Highlighting}[]
\FunctionTok{library}\NormalTok{(tidyverse)}
\FunctionTok{library}\NormalTok{(palmerpenguins)}
\FunctionTok{library}\NormalTok{(here)}
\FunctionTok{library}\NormalTok{(janitor)}
\FunctionTok{library}\NormalTok{(ggplot2)}
\FunctionTok{library}\NormalTok{(ragg)}
\FunctionTok{library}\NormalTok{(svglite)}
\FunctionTok{library}\NormalTok{(tinytex)}
\FunctionTok{library}\NormalTok{(yaml)}
\end{Highlighting}
\end{Shaded}

Loading the data:

\begin{Shaded}
\begin{Highlighting}[]
\FunctionTok{head}\NormalTok{(penguins\_raw)}
\end{Highlighting}
\end{Shaded}

\begin{verbatim}
## # A tibble: 6 x 17
##   studyName `Sample Number` Species          Region Island Stage `Individual ID`
##   <chr>               <dbl> <chr>            <chr>  <chr>  <chr> <chr>          
## 1 PAL0708                 1 Adelie Penguin ~ Anvers Torge~ Adul~ N1A1           
## 2 PAL0708                 2 Adelie Penguin ~ Anvers Torge~ Adul~ N1A2           
## 3 PAL0708                 3 Adelie Penguin ~ Anvers Torge~ Adul~ N2A1           
## 4 PAL0708                 4 Adelie Penguin ~ Anvers Torge~ Adul~ N2A2           
## 5 PAL0708                 5 Adelie Penguin ~ Anvers Torge~ Adul~ N3A1           
## 6 PAL0708                 6 Adelie Penguin ~ Anvers Torge~ Adul~ N3A2           
## # i 10 more variables: `Clutch Completion` <chr>, `Date Egg` <date>,
## #   `Culmen Length (mm)` <dbl>, `Culmen Depth (mm)` <dbl>,
## #   `Flipper Length (mm)` <dbl>, `Body Mass (g)` <dbl>, Sex <chr>,
## #   `Delta 15 N (o/oo)` <dbl>, `Delta 13 C (o/oo)` <dbl>, Comments <chr>
\end{verbatim}

\begin{Shaded}
\begin{Highlighting}[]
\FunctionTok{colnames}\NormalTok{(penguins\_raw)}
\end{Highlighting}
\end{Shaded}

\begin{verbatim}
##  [1] "studyName"           "Sample Number"       "Species"            
##  [4] "Region"              "Island"              "Stage"              
##  [7] "Individual ID"       "Clutch Completion"   "Date Egg"           
## [10] "Culmen Length (mm)"  "Culmen Depth (mm)"   "Flipper Length (mm)"
## [13] "Body Mass (g)"       "Sex"                 "Delta 15 N (o/oo)"  
## [16] "Delta 13 C (o/oo)"   "Comments"
\end{verbatim}

\begin{Shaded}
\begin{Highlighting}[]
\FunctionTok{write.csv}\NormalTok{(penguins\_raw, }\FunctionTok{here}\NormalTok{(}\StringTok{"data"}\NormalTok{, }\StringTok{"penguins\_raw.csv"}\NormalTok{))}
\end{Highlighting}
\end{Shaded}

Cleaning the data:

\begin{Shaded}
\begin{Highlighting}[]
\NormalTok{penguins\_clean }\OtherTok{\textless{}{-}}\NormalTok{ penguins\_raw }\SpecialCharTok{\%\textgreater{}\%}
\FunctionTok{select}\NormalTok{(}\SpecialCharTok{{-}}\NormalTok{Comments) }\SpecialCharTok{\%\textgreater{}\%}
\FunctionTok{select}\NormalTok{(}\SpecialCharTok{{-}}\FunctionTok{starts\_with}\NormalTok{(}\StringTok{"Delta"}\NormalTok{)) }\SpecialCharTok{\%\textgreater{}\%}
\FunctionTok{clean\_names}\NormalTok{() }\SpecialCharTok{\%\textgreater{}\%}
\FunctionTok{mutate}\NormalTok{(}\AttributeTok{species =} \FunctionTok{gsub}\NormalTok{(}\StringTok{" .*"}\NormalTok{, }\StringTok{""}\NormalTok{, species))  }\CommentTok{\# Shorten species names}

\FunctionTok{colnames}\NormalTok{(penguins\_clean)}
\end{Highlighting}
\end{Shaded}

\begin{verbatim}
##  [1] "study_name"        "sample_number"     "species"          
##  [4] "region"            "island"            "stage"            
##  [7] "individual_id"     "clutch_completion" "date_egg"         
## [10] "culmen_length_mm"  "culmen_depth_mm"   "flipper_length_mm"
## [13] "body_mass_g"       "sex"
\end{verbatim}

Saving the clean data:

\begin{Shaded}
\begin{Highlighting}[]
\FunctionTok{write\_csv}\NormalTok{(penguins\_clean, }\FunctionTok{here}\NormalTok{(}\StringTok{"data"}\NormalTok{, }\StringTok{"penguins\_clean.csv"}\NormalTok{))}
\end{Highlighting}
\end{Shaded}

\subsection{QUESTION 01: Data Visualisation for Science
Communication}\label{question-01-data-visualisation-for-science-communication}

\subsubsection{a) Figure that is correct but badly communicates the
data:}\label{a-figure-that-is-correct-but-badly-communicates-the-data}

\includegraphics{RepFiguresAssignment_files/figure-latex/bad figure code-1.pdf}

\subsubsection{b) How do the design choices mislead the reader about the
underlying data? (100-300
words)}\label{b-how-do-the-design-choices-mislead-the-reader-about-the-underlying-data-100-300-words}

A scatter plot can be a useful exploratory figure to convey the
relationship between body mass and flipper length in the palmerpenguins
dataset. However, there are some design choices that make this
particular scatter plot misleading and prevent it from conveying
important information contained in the dataset.

\begin{enumerate}
\def\labelenumi{\arabic{enumi}.}
\item
  The high geom\_point size obscures patterns in the data by making
  points severely overlap.
\item
  The alpha value of 1 makes the points completely opaque, once again
  obscuring patterns in the data by making overlapping points hard to
  discern.
\item
  The random colors chosen for the species and a lack of a legend
  prevents the reader from knowing which penguin species is represented
  by which colors. This hinders the figure from communicating important
  information about species-specific trends in the relationship between
  body mass and flipper length.
\end{enumerate}

With a smaller geom\_point size, a smaller alpha value, and a proper
legend with appropriate species-specific colors, this scatter plot would
be very effective in conveying how flipper length changes with
increasing body mass for the three penguin species Adelie (Pygoscelis
adeliae), Chinstrap (Pygoscelis antarcticus), and Gentoo (Pygoscelis
papua).

\begin{center}\rule{0.5\linewidth}{0.5pt}\end{center}

\subsection{QUESTION 2: Data Pipeline}\label{question-2-data-pipeline}

\begin{Shaded}
\begin{Highlighting}[]
\CommentTok{\# Loading libraries}
\FunctionTok{library}\NormalTok{(tidyverse)}
\FunctionTok{library}\NormalTok{(palmerpenguins)}
\FunctionTok{library}\NormalTok{(here)}
\FunctionTok{library}\NormalTok{(janitor)}
\FunctionTok{library}\NormalTok{(ggplot2)}
\FunctionTok{library}\NormalTok{(ragg)}
\FunctionTok{library}\NormalTok{(svglite)}
\FunctionTok{library}\NormalTok{(tinytex)}
\FunctionTok{library}\NormalTok{(yaml)}
\end{Highlighting}
\end{Shaded}

\subsubsection{Introduction}\label{introduction}

Species occupy distinct ecological niches in a community which promote
coexistence. One of the key contributors to niche differences between
sympatric species are diet and foraging habits, adaptations for which
are often reflected in the beak morphology in birds.

This analysis utilizes data collected from three largely sympatric
penguin species - Adelie (Pygoscelis adeliae), Chinstrap (Pygoscelis
antarcticus), and Gentoo (Pygoscelis papua) -to investigate whether
there is a significant difference in beak length between them.
Differences in beak length can be linked to wider differences in the
beak morphology and are likely to reflect differences in feeding ecology
between species. Insights from this analysis can contribute towards an
increased understanding of these species' ecologies, which is important
when making predictions about their future habitat and survival.

The data for this analysis is contained in the palmerpenguins dataset.
The dataset was first loaded into RStudio, then cleaned to ensure
consistency in the names of rows and columns, shorten species names, and
remove empty rows and columns.

\begin{Shaded}
\begin{Highlighting}[]
\CommentTok{\# Loading the data}
\FunctionTok{library}\NormalTok{(palmerpenguins)}
\FunctionTok{head}\NormalTok{(penguins\_raw)}
\end{Highlighting}
\end{Shaded}

\begin{verbatim}
## # A tibble: 6 x 17
##   studyName `Sample Number` Species          Region Island Stage `Individual ID`
##   <chr>               <dbl> <chr>            <chr>  <chr>  <chr> <chr>          
## 1 PAL0708                 1 Adelie Penguin ~ Anvers Torge~ Adul~ N1A1           
## 2 PAL0708                 2 Adelie Penguin ~ Anvers Torge~ Adul~ N1A2           
## 3 PAL0708                 3 Adelie Penguin ~ Anvers Torge~ Adul~ N2A1           
## 4 PAL0708                 4 Adelie Penguin ~ Anvers Torge~ Adul~ N2A2           
## 5 PAL0708                 5 Adelie Penguin ~ Anvers Torge~ Adul~ N3A1           
## 6 PAL0708                 6 Adelie Penguin ~ Anvers Torge~ Adul~ N3A2           
## # i 10 more variables: `Clutch Completion` <chr>, `Date Egg` <date>,
## #   `Culmen Length (mm)` <dbl>, `Culmen Depth (mm)` <dbl>,
## #   `Flipper Length (mm)` <dbl>, `Body Mass (g)` <dbl>, Sex <chr>,
## #   `Delta 15 N (o/oo)` <dbl>, `Delta 13 C (o/oo)` <dbl>, Comments <chr>
\end{verbatim}

\begin{Shaded}
\begin{Highlighting}[]
\FunctionTok{colnames}\NormalTok{(penguins\_raw)}
\end{Highlighting}
\end{Shaded}

\begin{verbatim}
##  [1] "studyName"           "Sample Number"       "Species"            
##  [4] "Region"              "Island"              "Stage"              
##  [7] "Individual ID"       "Clutch Completion"   "Date Egg"           
## [10] "Culmen Length (mm)"  "Culmen Depth (mm)"   "Flipper Length (mm)"
## [13] "Body Mass (g)"       "Sex"                 "Delta 15 N (o/oo)"  
## [16] "Delta 13 C (o/oo)"   "Comments"
\end{verbatim}

\begin{Shaded}
\begin{Highlighting}[]
\FunctionTok{write.csv}\NormalTok{(penguins\_raw, }\FunctionTok{here}\NormalTok{(}\StringTok{"data"}\NormalTok{, }\StringTok{"penguins\_raw.csv"}\NormalTok{))}

\CommentTok{\# Cleaning the data}
\NormalTok{penguins\_clean }\OtherTok{\textless{}{-}}\NormalTok{ penguins\_raw }\SpecialCharTok{\%\textgreater{}\%}
\FunctionTok{select}\NormalTok{(}\SpecialCharTok{{-}}\NormalTok{Comments) }\SpecialCharTok{\%\textgreater{}\%}
\FunctionTok{select}\NormalTok{(}\SpecialCharTok{{-}}\FunctionTok{starts\_with}\NormalTok{(}\StringTok{"Delta"}\NormalTok{)) }\SpecialCharTok{\%\textgreater{}\%}
\FunctionTok{clean\_names}\NormalTok{() }\SpecialCharTok{\%\textgreater{}\%}
\FunctionTok{mutate}\NormalTok{(}\AttributeTok{species =} \FunctionTok{gsub}\NormalTok{(}\StringTok{" .*"}\NormalTok{, }\StringTok{""}\NormalTok{, species))  }\CommentTok{\# Shorten species names}

\FunctionTok{colnames}\NormalTok{(penguins\_clean)}
\end{Highlighting}
\end{Shaded}

\begin{verbatim}
##  [1] "study_name"        "sample_number"     "species"          
##  [4] "region"            "island"            "stage"            
##  [7] "individual_id"     "clutch_completion" "date_egg"         
## [10] "culmen_length_mm"  "culmen_depth_mm"   "flipper_length_mm"
## [13] "body_mass_g"       "sex"
\end{verbatim}

Following cleaning, the data was subsetted to only include culmen
length, culmen depth (analagous to bill length and depth), and species
information to explore bill dimensions in Adelie, Chinstrap, and Gentoo
penguins.

\begin{Shaded}
\begin{Highlighting}[]
\CommentTok{\# Subsetting the columns culmen length, culmen depth, species and removing NAs}
\NormalTok{penguins\_data\_for\_scatterplot }\OtherTok{\textless{}{-}}\NormalTok{ penguins\_clean }\SpecialCharTok{\%\textgreater{}\%}
  \FunctionTok{select}\NormalTok{(culmen\_length\_mm, culmen\_depth\_mm, species) }\SpecialCharTok{\%\textgreater{}\%}
  \FunctionTok{na.omit}\NormalTok{(culmen\_length\_mm, culmen\_depth\_mm)}
\end{Highlighting}
\end{Shaded}

\subsubsection{Hypothesis}\label{hypothesis}

There is a statistically significant difference in the mean beak lengths
of Adelie, Chinstrap, and Gentoo penguins.

\subsubsection{Statistical Methods}\label{statistical-methods}

This analysis used ANOVA to test for differences between the means of
the three groups (species) and performed a post-hoc Tukey HSD test to
investigate pairwise differences in mean beak lengths.

\begin{Shaded}
\begin{Highlighting}[]
\CommentTok{\# One{-}way ANOVA}
\NormalTok{anova\_result }\OtherTok{\textless{}{-}} \FunctionTok{aov}\NormalTok{(culmen\_length\_mm }\SpecialCharTok{\textasciitilde{}}\NormalTok{ species, }\AttributeTok{data =}\NormalTok{ penguins\_data\_for\_scatterplot)}

\CommentTok{\# Post{-}hoc Tukey HSD test for pairwise comparisons}
\NormalTok{tukey\_result }\OtherTok{\textless{}{-}} \FunctionTok{TukeyHSD}\NormalTok{(anova\_result)}
\end{Highlighting}
\end{Shaded}

\subsubsection{Results \& Discussion}\label{results-discussion}

A scatter plot of the raw data was created to explore bill dimensions in
the three penguin species.

\begin{Shaded}
\begin{Highlighting}[]
\CommentTok{\# Define color mapping with names for each species}
\NormalTok{species\_colours }\OtherTok{\textless{}{-}} \FunctionTok{c}\NormalTok{(}
  \StringTok{"Adelie"} \OtherTok{=} \StringTok{"darkorange"}\NormalTok{, }
  \StringTok{"Chinstrap"} \OtherTok{=} \StringTok{"purple"}\NormalTok{, }
  \StringTok{"Gentoo"} \OtherTok{=} \StringTok{"cyan4"}\NormalTok{)}

\CommentTok{\# Scatter plot of bill dimensions by species with custom colors}
\NormalTok{bill\_scatterplot }\OtherTok{\textless{}{-}} \FunctionTok{ggplot}\NormalTok{(penguins\_data\_for\_scatterplot, }\FunctionTok{aes}\NormalTok{(}\AttributeTok{x =}\NormalTok{ culmen\_length\_mm, }\AttributeTok{y =}\NormalTok{ culmen\_depth\_mm, }\AttributeTok{color =}\NormalTok{ species)) }\SpecialCharTok{+}
  \FunctionTok{geom\_point}\NormalTok{(}\AttributeTok{alpha =} \FloatTok{0.7}\NormalTok{, }\AttributeTok{size =} \DecValTok{3}\NormalTok{) }\SpecialCharTok{+}
  \FunctionTok{scale\_color\_manual}\NormalTok{(}\AttributeTok{values =}\NormalTok{ species\_colours) }\SpecialCharTok{+}
  \FunctionTok{labs}\NormalTok{(}
    \AttributeTok{title =} \StringTok{"Bill Dimensions by Penguin Species"}\NormalTok{,}
    \AttributeTok{x =} \StringTok{"Bill Length (mm)"}\NormalTok{,}
    \AttributeTok{y =} \StringTok{"Bill Depth (mm)"}\NormalTok{,}
    \AttributeTok{color =} \StringTok{"Species"}
\NormalTok{  ) }\SpecialCharTok{+}
  \FunctionTok{theme\_minimal}\NormalTok{()}

\CommentTok{\# Display the plot}
\FunctionTok{print}\NormalTok{(bill\_scatterplot)}
\end{Highlighting}
\end{Shaded}

\includegraphics{RepFiguresAssignment_files/figure-latex/unnamed-chunk-9-1.pdf}

The scatter plot shows how beak depth changes in relation to beak
length, and visual inspection of it reveals that the three species
occupy distinct areas of the plot, hinting at differences in beak
morphology between them. While Adelie penguins' beaks are shorter but
deeper, those of Gentoo penguins seem longer yet less deep. Chinstraps
occupy an area between these two species on the curve, with intermediate
beak length and depth.

The figure was appropriately scaled and saved.

\begin{Shaded}
\begin{Highlighting}[]
\CommentTok{\# Saving as .png}
\FunctionTok{agg\_png}\NormalTok{(}\StringTok{"figures/bill\_scatterplot.png"}\NormalTok{, }
        \AttributeTok{width =} \DecValTok{40}\NormalTok{, }\AttributeTok{height =} \DecValTok{40}\NormalTok{, }\AttributeTok{units =} \StringTok{"cm"}\NormalTok{, }\AttributeTok{res =} \DecValTok{300}\NormalTok{, }\AttributeTok{scaling =} \DecValTok{2}\NormalTok{)}
\FunctionTok{print}\NormalTok{(bill\_scatterplot)}
\FunctionTok{dev.off}\NormalTok{()}
\end{Highlighting}
\end{Shaded}

\begin{verbatim}
## pdf 
##   2
\end{verbatim}

\begin{Shaded}
\begin{Highlighting}[]
\CommentTok{\# Saving as .svg}
\NormalTok{inches\_conversion }\OtherTok{=} \FloatTok{2.54}
\FunctionTok{svglite}\NormalTok{(}\StringTok{"figures/bill\_scatterplot.svg"}\NormalTok{, }
        \AttributeTok{width =} \DecValTok{40} \SpecialCharTok{/}\NormalTok{ inches\_conversion, }
        \AttributeTok{height =} \DecValTok{40} \SpecialCharTok{/}\NormalTok{ inches\_conversion, }
        \AttributeTok{scaling =} \DecValTok{2}\NormalTok{)}
\FunctionTok{print}\NormalTok{(bill\_scatterplot)}
\FunctionTok{dev.off}\NormalTok{()}
\end{Highlighting}
\end{Shaded}

\begin{verbatim}
## pdf 
##   2
\end{verbatim}

Next, an ANOVA was conducted to test for differences in mean beak depths
between the three penguin species, after which the Tukey HSD test was
performed to investigate pairwise comparisons.

\begin{Shaded}
\begin{Highlighting}[]
\FunctionTok{summary}\NormalTok{(anova\_result)}
\end{Highlighting}
\end{Shaded}

\begin{verbatim}
##              Df Sum Sq Mean Sq F value Pr(>F)    
## species       2   7194    3597   410.6 <2e-16 ***
## Residuals   339   2970       9                   
## ---
## Signif. codes:  0 '***' 0.001 '**' 0.01 '*' 0.05 '.' 0.1 ' ' 1
\end{verbatim}

\begin{Shaded}
\begin{Highlighting}[]
\FunctionTok{print}\NormalTok{(tukey\_result)}
\end{Highlighting}
\end{Shaded}

\begin{verbatim}
##   Tukey multiple comparisons of means
##     95% family-wise confidence level
## 
## Fit: aov(formula = culmen_length_mm ~ species, data = penguins_data_for_scatterplot)
## 
## $species
##                       diff       lwr        upr     p adj
## Chinstrap-Adelie 10.042433  9.024859 11.0600064 0.0000000
## Gentoo-Adelie     8.713487  7.867194  9.5597807 0.0000000
## Gentoo-Chinstrap -1.328945 -2.381868 -0.2760231 0.0088993
\end{verbatim}

The ANOVA results indicate that there is a statistically significant
difference in bill length across the three penguin species. This is due
to the highly significant p-value at the 0.001 level. This result is
consistent with the hypothesis that penguin species have differing bill
lengths, likely due to species-specific ecological adaptations.

All p-values from the Tukey test are \textless0.05, indicating the
significant difference in bill length in all pairwise species
comparisons. Additionally, the 95\% confidence intervals for the
differences do not include 0 for any pair. Taken together, the results
of the ANOVA and Tukey test suggest a statistically significant
difference in the bill lengths between all three penguin species in the
analysis, supporting the hypothesis. The differences in bill length
likely reflect ecological adaptations of the penguin species to their
respective diet and foraging strategies.

Following this, an interval plot was produced which visualizes the
differences in mean beak length for all pairwise comparisons, including
their relative statistical significance taken from the Tukey test.

Before producing the plot, results of the Tukey HSD test were extracted
into a data frame, to which a species comparison column and indication
of significance levels were added.

\begin{Shaded}
\begin{Highlighting}[]
\CommentTok{\# Extracting the Tukey HSD results into a dataframe:}
\NormalTok{tukey\_df }\OtherTok{\textless{}{-}} \FunctionTok{as.data.frame}\NormalTok{(tukey\_result}\SpecialCharTok{$}\NormalTok{species)}

\CommentTok{\# Adding a column for the species comparisons}
\NormalTok{tukey\_df}\SpecialCharTok{$}\NormalTok{Comparison }\OtherTok{\textless{}{-}} \FunctionTok{rownames}\NormalTok{(tukey\_df)}

\CommentTok{\# Creating the pairwise\_diff data frame with significance levels indicated}
\NormalTok{pairwise\_diff }\OtherTok{\textless{}{-}}\NormalTok{ tukey\_df }\SpecialCharTok{\%\textgreater{}\%}
  \FunctionTok{select}\NormalTok{(Comparison, diff, lwr, upr, }\StringTok{\textasciigrave{}}\AttributeTok{p adj}\StringTok{\textasciigrave{}}\NormalTok{) }\SpecialCharTok{\%\textgreater{}\%}
  \FunctionTok{rename}\NormalTok{(}
    \AttributeTok{Difference =}\NormalTok{ diff,}
    \AttributeTok{Lower =}\NormalTok{ lwr,}
    \AttributeTok{Upper =}\NormalTok{ upr,}
    \AttributeTok{p\_value =} \StringTok{\textasciigrave{}}\AttributeTok{p adj}\StringTok{\textasciigrave{}}
\NormalTok{  )  }\SpecialCharTok{\%\textgreater{}\%}
  \FunctionTok{mutate}\NormalTok{(}
    \AttributeTok{significance =} \FunctionTok{case\_when}\NormalTok{(}
\NormalTok{      p\_value }\SpecialCharTok{\textless{}} \FloatTok{0.001} \SpecialCharTok{\textasciitilde{}} \StringTok{"***"}\NormalTok{,}
\NormalTok{      p\_value }\SpecialCharTok{\textless{}} \FloatTok{0.01} \SpecialCharTok{\textasciitilde{}} \StringTok{"**"}\NormalTok{,}
\NormalTok{      p\_value }\SpecialCharTok{\textless{}} \FloatTok{0.05} \SpecialCharTok{\textasciitilde{}} \StringTok{"*"}\NormalTok{,}
      \ConstantTok{TRUE} \SpecialCharTok{\textasciitilde{}} \StringTok{"ns"}  \CommentTok{\# Not significant}
\NormalTok{    ) )}

\CommentTok{\# Print the resulting pairwise\_diff data frame}
\FunctionTok{print}\NormalTok{(pairwise\_diff)}
\end{Highlighting}
\end{Shaded}

\begin{verbatim}
##                        Comparison Difference     Lower      Upper     p_value
## Chinstrap-Adelie Chinstrap-Adelie  10.042433  9.024859 11.0600064 0.000000000
## Gentoo-Adelie       Gentoo-Adelie   8.713487  7.867194  9.5597807 0.000000000
## Gentoo-Chinstrap Gentoo-Chinstrap  -1.328945 -2.381868 -0.2760231 0.008899333
##                  significance
## Chinstrap-Adelie          ***
## Gentoo-Adelie             ***
## Gentoo-Chinstrap           **
\end{verbatim}

\begin{Shaded}
\begin{Highlighting}[]
\CommentTok{\# Interval plot with significance annotations}
\NormalTok{interval\_plot }\OtherTok{\textless{}{-}} \FunctionTok{ggplot}\NormalTok{(pairwise\_diff, }\FunctionTok{aes}\NormalTok{(}\AttributeTok{x =}\NormalTok{ Comparison, }\AttributeTok{y =}\NormalTok{ Difference)) }\SpecialCharTok{+}
  \FunctionTok{geom\_point}\NormalTok{(}\AttributeTok{size =} \DecValTok{4}\NormalTok{, }\AttributeTok{color =} \StringTok{"maroon2"}\NormalTok{) }\SpecialCharTok{+}  \CommentTok{\# Mean differences as points}
  \FunctionTok{geom\_errorbar}\NormalTok{(}\FunctionTok{aes}\NormalTok{(}\AttributeTok{ymin =}\NormalTok{ Lower, }\AttributeTok{ymax =}\NormalTok{ Upper), }\AttributeTok{width =} \FloatTok{0.2}\NormalTok{, }\AttributeTok{color =} \StringTok{"black"}\NormalTok{) }\SpecialCharTok{+} \CommentTok{\# Vertical error bars}
  \FunctionTok{geom\_hline}\NormalTok{(}\AttributeTok{yintercept =} \DecValTok{0}\NormalTok{, }\AttributeTok{linetype =} \StringTok{"dashed"}\NormalTok{, }\AttributeTok{color =} \StringTok{"navyblue"}\NormalTok{, }\AttributeTok{linewidth =} \FloatTok{0.8}\NormalTok{) }\SpecialCharTok{+}  \CommentTok{\# Zero difference reference line}
  \FunctionTok{geom\_text}\NormalTok{(}
    \FunctionTok{aes}\NormalTok{(}\AttributeTok{label =}\NormalTok{ significance, }\AttributeTok{y =}\NormalTok{ Upper }\SpecialCharTok{+} \FloatTok{0.5}\NormalTok{),  }\CommentTok{\# Positioning significance levels above the error bars}
    \AttributeTok{size =} \DecValTok{5}\NormalTok{,}
    \AttributeTok{color =} \StringTok{"black"}
\NormalTok{  ) }\SpecialCharTok{+}
  \FunctionTok{labs}\NormalTok{(}\AttributeTok{title =} \StringTok{"Pairwise Differences in Bill Length Between Penguin Species"}\NormalTok{,}
    \AttributeTok{x =} \StringTok{"Species Comparison"}\NormalTok{,}
    \AttributeTok{y =} \StringTok{"Difference in Mean Bill Length (mm)"}\NormalTok{) }\SpecialCharTok{+}
  \FunctionTok{theme\_bw}\NormalTok{()}

\CommentTok{\# Display the plot}
\FunctionTok{print}\NormalTok{(interval\_plot)}
\end{Highlighting}
\end{Shaded}

\includegraphics{RepFiguresAssignment_files/figure-latex/unnamed-chunk-13-1.pdf}

The interval plot highlights the statistical significance of all
pairwise comparisons and shows that the 95\% confidence interval does
not overlap 0 for any pair. From this figure, it can be concluded that
the average beak length is highest for Chinstrap penguins followed by
Gentoo. Adelie penguins have the shortest beaks between the three
species.

The significant differences in the mean bill lengths of the three
penguin species likely reflect ecological adaptations to different
feeding niches. Adelie penguins feed largely on fish and krill, and
usually catch their prey near the surface of the water. Like Adelie
penguins, Chinstrap penguins also have a diet consisting of fish and
krill (Trivelpiece et al., 1987). Both of these species feed offshore
and are shallow divers, although Chinstraps tend to dive deeper than
Adelie penguins. Gentoos have a more diverse generalist diet consisting
of squid, fish, and crustaceans. They feed inshore and dive to depths of
up to 200 m to catch their prey. Their ability to dive deeper enables
them access to a feeding niche which is unavailable to the other two
species (Trivelpiece et al., 1987). Although these differences in diet
composition and foraging behaviour may seem minor, they can be enough to
result in distinct niches for these species to ensure their coexistence.
It is important to note that size differences exist between these
species, however the size order is not mirrored in the order of beak
length. Gentoos are the largest of these three species yet come second
in average beak length. This finding strengthens the idea that
differences in bill length in these species are partly a result of
differences in feeding ecology, although more detailed geometric
morphometric analyses paired with behavioural observations are needed
for further support.

\subsubsection{Conclusion}\label{conclusion}

This analysis aimed to test whether there is a statistically significant
difference in bill length between Adelie, Chinstrap, and Gentoo
penguins. The bill morphology of penguins is shaped by adaptations to
their feeding ecology since bills are used in the acquisition and
consumption of food items. The finding that the mean beak lengths differ
significantly between these three penguin species hints at potential
differences regarding their feeding niches. Additional factors to be
considered in further analyses include beak length relative to body
size, as well as other parameters characterizing beak morphology.

\subsubsection{References}\label{references}

\begin{itemize}
\item
  A., L.~et al.~(2002) `Conflict or co-existence? foraging distribution
  and competition for prey between Adélie and chinstrap
  penguins.',~Marine Biology, 141(6), pp.~1165--1174.
  \url{doi:10.1007/s00227-002-0899-1}.~
\item
  Birds \& Seals of the Falkland Islands- Falklands birds \& seals -
  Falkland Islands wildlife guide. Available at:
  \url{http://www.seabirds.org/birds.htm} (Accessed: 11 December 2024).~
\item
  Oceanwide Expeditions. Available at:
  \url{https://oceanwide-expeditions.com/blog/meet-all-6-antarctic-penguin-species}
  (Accessed: 11 December 2024).~
\item
  Gentoo Penguins \textasciitilde{} Marinebio Conservation
  Society~(2023)~MarineBio Conservation Society. Available at:
  \url{https://www.marinebio.org/species/gentoo-penguins/pygoscelis-papua/}
  (Accessed: 11 December 2024).~
\item
  Trivelpiece, W.Z., Trivelpiece, S.G. and Volkman, N.J. (1987)
  `Ecological segregation of Adelie, Gentoo, and chinstrap penguins at
  king george island, Antarctica',~Ecology, 68(2), pp.~351--361.
  \url{doi:10.2307/1939266}.~
\end{itemize}

\begin{center}\rule{0.5\linewidth}{0.5pt}\end{center}

\subsection{Question 3: Open Science}\label{question-3-open-science}

\subsubsection{a) GitHub}\label{a-github}

\emph{Upload your RProject you created for \textbf{Question 2} and any
files and subfolders used to GitHub. Do not include any identifiers such
as your name. Make sure your GitHub repo is public.}

\emph{GitHub link:}

\url{https://github.com/biouser2/Reproducable-Science-Assignment}

\subsubsection{b) Share your repo with a partner, download, and try to
run their data
pipeline.}\label{b-share-your-repo-with-a-partner-download-and-try-to-run-their-data-pipeline.}

\emph{Partner's GitHub link:}

\url{https://github.com/Biology11/Reproducible-Science-and-Figures}

\subsubsection{c) Reflect on your experience running their code.
(300-500
words)}\label{c-reflect-on-your-experience-running-their-code.-300-500-words}

Running my partner's code was mostly a straightforward process. The
analysis is structured clearly and well. The use of headings and
comments made it easy to understand the function of each code chunk,
even if the specifics of the code were not familiar to me. The necessary
libraries were loaded and my partner's code ran without needing to fix
anything. The use of intuitive and descriptive variable names, such as
``penguins\_clean'' and ``adelie\_penguins'', made the code easy to read
for a human and not just for a computer. It would have been good to save
datasets such as ``penguins\_clean'' as .csv files for easy access in
the future. The use of piping to clean the data contributed to the
reproducibility of the code as well as making it easy to alter the code
for different analysis needs in the future. The line of code where the
column name ``Sex'' is modified to not be capitalized could have been
included in the data cleaning pipe as well; this would overcome the
confusion created by having variables called ``adelie\_penguins'' as
well as adelie\_penguins\$sex.~ Additionally, creating a function in R
would have made these steps even more reproducible. If my partner were
to create a ``cleaning function'' for example, it could be possible to
use the same code for a different dataset, making their analysis
replicable.~

The figures produced in the analysis are very clear in communicating the
relevant information from the dataset. The transparency of the bins of
the histogram are useful in showing the distribution for males and
females in the area of overlap. For some of the less intuitive elements
of the code, including a comment next to certain functions could improve
clarity for users. For example, it could be useful to add a comment
after the annotate(``text'',\ldots) line when plotting the histogram to
explain what this line of code does. Additionally, writing code to
display the mean of each distribution on the figure as a dot, perhaps
indicating the level of significance between the means with asterisks on
the figure as well, would make the plot more efficient in conveying the
relevant information. However even in this step, the code my partner
wrote is reproducible since it does not directly include a number for
the mean flipper length in the code and instead uses columns in the two
sample t-test results table. One way of making the figures easier to
manipulate for other users is by defining a color vector for both sexes
beforehand and including the color vector in the code instead of
manually assigning colors. Despite minor suggestions, the code and
analysis runs without problem and is effective in communicating
important information, as well as being largely reproducible.

\subsubsection{d) Reflect on your own code based on your experience with
your partner's code and their review of yours. (300-500
words)}\label{d-reflect-on-your-own-code-based-on-your-experience-with-your-partners-code-and-their-review-of-yours.-300-500-words}

The exercise of trying to run my partner's code and get their feedback
on running mine was a very informative process from which I have gained
valuable insights into writing reproducible code. An improvement my
partner suggested to me was creating a function for the data cleaning
and subsetting stages, which would improve the reproducibility of my
code and could make it be used on different datasets seamlessly. This
was very noteworthy to me as my main suggestion to my partner's code was
also the same: creating functions for steps such as data cleaning. It
has now become clear to me that while one may think their line of code
is well annotated and thus reproducible, creating functions where they
are useful enables research to go an extra step farther in terms of
reproducibility. Based on personal experience, as well as the literature
on the barriers to reproducibility in science, I believe what prevents
many people from writing their own functions in R is a lack of
experience with such practices. However, considering all of the
information available online which can aid one in writing their own
functions, as well as their utility in making research more
reproducible, I can say that this is something I will look to implement
in my future coding projects.~

Another comment I received was that my partner had difficulty running my
code to save figures, and that this part of my code should be more
thoroughly annotated. I agree with this suggestion and would ensure
annotations are present in all less intuitive elements of my code for
the future. Based on my experience with my partner's code, I can see
that using annotations where necessary can be very useful in making code
accessible and understandable even to people who are not familiar with
the methods used in the analysis. Some positive feedbacks I got included
the use of species-specific color vectors for figures. This was an area
in which I suggested improvements to my partner's code as it ensures
consistency amongst the figures as well as aiding in reproducibility.
Overall, this exercise has demonstrated the importance of using best
practices when coding to ensure reproducibility, and elements which are
important for this include writing functions where appropriate,
annotating the code for elements which are less intuitive, writing
intuitive code to save datasets and plots, as well as defining color
vectors to be used in figures.~

\subsubsection{e) What are the main barriers for scientists to share
their data and code, and what could be done to overcome them? (500-700
words)}\label{e-what-are-the-main-barriers-for-scientists-to-share-their-data-and-code-and-what-could-be-done-to-overcome-them-500-700-words}

\hfill\break
Despite the demonstrated benefits of data and code sharing, as well as
the importance of this practice for the advancement of science, its
wider adoption is hindered by certain barriers. For instance,
researchers may fear being ``scooped'', where another researcher
completes an analysis the author had intended to do themselves (Gomes et
al., 2022). The time required to clean one's code to ensure it is
reproducible and understandable to others presents another barrier to
the dissemination of code. Scientists may fear that if they make their
data and code public, others may find errors in them which may hinder
their career progress. Additionally, many researchers might simply not
know best practices to ensure the reproducibility of their analyses, as
well as how to make their data and code public (Gomes et al., 2022).

It is important to acknowledge that data and code sharing are not only
beneficial for the advancement of science by reducing the time
researchers spend rewriting code which could otherwise be obtained
through sharing, but also for the individual scientist themselves. In
fact, various studies have demonstrated the increased citation rate of
papers as a result of sharing their data and code (McKiernan et al.,
2016). It is important for researchers to consider that their rights to
ownership of data or code can be protected by appropriate licensing, and
that the opportunity to publish pre-prints allows them to disseminate
their ideas rapidly and claim the projects they want to conduct (Gomes
et al., 2022). To overcome fears of inadequate code, it is essential
that researchers familiarize themselves with best practices in coding,
which will not only allow them to engage in reproducible science, but
also make it easier for themselves to rerun analyses in the future or do
further work on a given project. One study aimed to establish the
reproducibility of research uploaded onto the Harvard Database
repository, and found that they were able to reduce the percentage of R
files that produced an error when running from 74\% to 56\% by
implementing a simple code cleaning procedure (Trisovic et al., 2022).
Their analysis showed that the reproducibility of research may be
increased through the use of literate programming, for example by adding
comments to explain code chunks (Trisovic et al., 2022). It was noted,
however, that best practices in this regard suggest minimizing the
number of comments to those necessary, and implementing intuitive names
for functions and variables remains essential for reproducibility. The
use of formats such as RMarkdown allow ease of use of the code, which
also improves reproducibility. The authors frequently observed errors
due to the incompatibility of newer versions of R libraries with the
initial analysis, and strongly recommend an increased adoption of the
renv package in R which documents the libraries and versions used in an
analysis (Trisovic et al., 2022). Although researchers might not be used
to this practice, its wider adoption holds the potential to
significantly improve the reproducibility of analyses.~

Lastly, responsibility falls on journals and funding bodies to
incentivize data and code sharing. Although some funding institutions,
like the NIH in the United States, have been mandating data sharing for
decades, there is still the need for more organizations to adopt such
policies (Gomes et al., 2022). Although some journals mandate or
strongly encourage~ the sharing of data and code, there is scope for
significant improvement in this area, especially regarding the
effectiveness of the implementation of these policies. Journals should
provide links to the repository where the data and code in an analysis
is stored, and could even employ code editors as part of the peer review
process~ (Maitner et al., 2023). As for individuals, the sharing of data
and code as well as adoption of open science in general, is expected to
provide benefits to journals owing to an increase in citation rates and
impact.

\subsubsection{References}\label{references-1}

\begin{itemize}
\tightlist
\item
  Maitner, B.~\emph{et al.}~(2023)~\emph{Code sharing increases
  citations, but remains uncommon}~{[}Preprint{]}.
  \url{doi:10.21203/rs.3.rs-3222221/v1}.~
\item
  Trisovic, A.~\emph{et al.}~(2022) `A large-scale study on research
  code quality and execution',~\emph{Scientific Data}, 9(1).
  \url{doi:10.1038/s41597-022-01143-6}.~
\item
  A Rock-Star Researcher Spun a Web of Lies---and Nearly Got Away with
  It.
  \url{https://thewalrus.ca/a-rock-star-researcher-spun-a-web-of-lies-and-nearly-got-away-with-it/}
\item
  Gomes, D.G.~\emph{et al.}~(2022) `Why don't we share data and code?
  perceived barriers and benefits to public archiving
  practices',~\emph{Proceedings of the Royal Society B: Biological
  Sciences}, 289(1987). \url{doi:10.1098/rspb.2022.1113}.~
\item
  McKiernan, E.C.~\emph{et al.}~(2016a) `How open science helps
  researchers succeed',~\emph{eLife}, 5. \url{doi:10.7554/elife.16800}.~
\item
  Baker, M. (2016) `1,500 scientists lift the lid on
  reproducibility',~\emph{Nature}, 533(7604), pp.~452--454.
  \url{doi:10.1038/533452a}.~
\end{itemize}

\end{document}
